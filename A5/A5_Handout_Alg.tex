\documentclass[10pt,twocolumn]{article}
\usepackage[margin=0.75in]{geometry}                % See geometry.pdf to learn the layout options. There are lots.
\geometry{letterpaper}                   % ... or a4paper or a5paper or ... 
%\geometry{landscape}                % Activate for for rotated page geometry
%\usepackage[parfill]{parskip}    % Activate to begin paragraphs with an empty line rather than an indent
\setlength{\columnsep}{1cm}
\usepackage{graphicx}
\usepackage{amssymb}
\usepackage{epstopdf}
\usepackage{fullpage}
\usepackage[usenames]{color}
\usepackage{titlesec}
\usepackage{hyperref}
\usepackage{framed}
\usepackage{algorithmic}

\definecolor{light-gray}{gray}{0.45}
\titleformat{\section}
{\color{black}\normalfont\Large\bfseries}
{\color{black}\thesection}{1em}{}

\titleformat{\subsection}
{\color{light-gray}\normalfont\large\bfseries}
{\color{light-gray}\thesubsection}{1em}{}

\DeclareGraphicsRule{.tif}{png}{.png}{`convert #1 `dirname #1`/`basename #1 .tif`.png}

\title{\Huge{\bf Algorithm 5: Ray}}
\author{Comp175: Introduction to Computer Graphics -- Spring 2015}
\date{Algorithm due:  {\bf Monday April 20th} at 11:59pm}                                         % Activate to display a given date or no date

\begin{document}
\maketitle
%\section{}
%\subsection{}

\begin{verbatim}
Your Names: _____________________________

            _____________________________

Your CS Logins: _________________________

                _________________________
\end{verbatim}

\section{Instructions}
Complete this assignment only with your teammate. When a numerical answer is required, provide a reduced fraction (i.e. 1/3) or at least three decimal places (i.e. 0.333). Show all work; write your answers on this sheet. This algorithm handout is worth 3\% of your final grade for the class.

\begin{framed}
\noindent{\bf[2 points]} The high-level view of our ray tracer is exactly the same as for intersect, except for a few additions. Below is the high-level pseudocode for {\tt Intersect}. What needs to be changed/added to make this a full-fledged raytracer? Just specify what changes need to be made {no pseudocode please.}

\footnotesize{
\begin{algorithmic}
\FOR{$point \in Canvas$}
	\STATE Cast a ray to find the nearest object
	\IF{ray intersects an object}
		\FOR{each light}
			\STATE Cast a ray to the light and evaluate the lighting equation\\
			\STATE $Canvas[pt] = Canvas[pt]+ color$ with only diffuse/ambient components\\
		\ENDFOR
	\ELSE 
		\STATE $Canvas[pt] =$ background color
	\ENDIF
\ENDFOR
\end{algorithmic}
}

\vskip 5em
\end{framed}

\begin{framed}
\noindent{\bf[2 points]} Given a vector $\vec{v}$ and a surface normal $\vec{n}$, find the equation for the vector $\vec{r}$ which is the reflection of $\vec{v}$ about $\vec{n}$ (i.e. in the equal and opposite direction). Write your equation in terms of vector operations. How do you compute the color contributed by the reflected ray? Give a brief description.
\includegraphics[width=2.8in]{reflection.png}
\vskip 13em
\end{framed}

\begin{framed}
\noindent{\bf[1 point]} Is ray tracing a local or global illumination algorithm? Why?
\vskip 12em
\end{framed}

\begin{framed}
\noindent{\bf[1 point]} For a particular ray that intersects with an object, when do you not consider contribution from a given light source? How do you computationally determine when this scenario occurs?
\vskip 14em
\end{framed}

\begin{framed}
\noindent{\bf[2 points]} Recall that we can think of texture mapping in two steps. First, mapping from the object to the unit square, and second, mapping from the unit square to the texture map. Let $a$ and $b$ be the $x$ and $y$ values in the unit square that a particular point on an object gets mapped to in the first step. Note that $a$ and $b$ are calculated differently depending on the object. From here, how do you find the coordinates $(s, t)$ to look up in a texture map in terms of $a, b, u, v, w$ and $h$, where $u$ and $v$ are the number of repetitions in the $x$ and $y$
directions, respectively, $w$ is the texture width, and $h$ is the texture height?
\vskip 12em
\end{framed}
\vskip 6em

\begin{framed}
\noindent{\bf[1 point]} How do you use the color from the texture map and the {\tt blend} value in the lighting equation?
\vskip 12em
\end{framed}

\begin{framed}
\noindent{\bf[1 point]}  What is the Phong lighting model used for? What is the purpose of its exponent?
\vskip 12em
\end{framed}

\section{How to Submit}

Hand in a PDF version of your solutions using the following command:
\begin{center}
 {\tt provide comp175 a5-alg}
 \end{center}
\end{document}  
